\documentclass[a4paper]{article}
\usepackage{html}

\usepackage{makeidx}

\makeindex

\title{And Now For Something Completly Different \\
- or -\\
How I Learned to Stop Worrying and Love Web Apps}
\author{Marco Baringer <mb@bese.it>}
\date{}

\newcommand{\ucw}{\textsc{UCW}}
\newcommand{\UCW}{\textsc{UnCommon Web}}
\newcommand{\sym}[1]{\textsc{#1}}

\begin{document}

\maketitle

\tableofcontents

\section{Introduction to the Introduction}

\UCW{} is web framework\footnote{or platform or library or toolkit or
  whatever you want to call it} which provides developers with a new
paradigm for developing web based applications. The goal of this
document is to explain what \ucw{} is, what isn't and how to write
applications with it.

We begin with a brief introduction to Common Lisp, continuations and
web applications in general. Then we delve into UnCommon Web and its
component based GUI toolkit and continuation based web flow control
(don't worry if this doesn't make any sense yet, it will
later). Finally we provide a small real world example.

\subsection{Why \ucw{}?}{Distinguishing Features}

According to a recent study there are 18,004 web frameworks available,
what's so special about \ucw{}?

\subsubsection{Common Lisp}

Yes, Common Lisp is a feature, not a bug. Common Lisp has been called
``the programmable programming language,'' what this means is that
whatever interesting and usefull paradigm is developed in the future
Common Lisp, and therefore \ucw{} will probably be able to absorb it
in a cleanly and without requiring anything other downloading a new
version of the library.

Common Lisp was the first ``write once, run anywhere'' language. The
Common Lisp standard provides the equivalent of avery high level
Virtual Machine and the Common Lisp implementers have done an
excellent job of providing cross-platform OS libraries for things such
as sockets, threads and file system access.

\subsubsection{Component Oriented UI}

Like functions allow us to abstract and reuse behaviour, like classes
allow us to abstract and reuse a concect, components allow us to
abstract and resuse graphical user interfaces. Components are regular
CLOS objects which represent the behaviour and the appearence of a
part of the UI.

\subsubsection{Continuations}

You probably first heard about \UCW{} in the context of something
called ``continuations.'' Simply put continuations are a way to
suspend the execution of a program and restart it later when (if) we
need to. When we look at web applications from the right angle
continuations suddenly become a very powerfull yet almost invisible
tool.

This is important so I'll repeat myself: While continuations are the
reason \ucw{} is so powerfull you never need to think about
continuations when developing \ucw{} apps.

\section{Getting Started}

Every \ucw{} application starts with a component which represents the
browser's window:

\begin{verbatim}
(defcomponent my-window (simple-window-component)
  ())
\end{verbatim}

This defines a class called my-window, and makes it a subclass of the
standrad ucw component simple-window-component. simple-window-component

\section{Components}

\ucw{} applications have a GUI. We use the term \textit{component} to
refer to a server side object which represents the GUI. Rarely will a
single component object be used for the entire GUI, generally we will
have multiple component tied together (we'll so how in just a second),
each component will be responsible for one specrific part of the
GUI. For example, we'll have a \textit{window} component which will
take up the entire browser window and set http headers, stylesheets
and include various javascript files, the window component will then
pass off to a navigation component which deals with setting up the
menus and the likns for moving around the application. The navigation
component will pass control off to a content component which will
dispaly the current content of the window. This content component may
use a tabbed-pane component or a form component. etc.

Every component has a state which is represented by the current values
of its slots (member variables or attributes in standard OO
parlance). Every component also has a set of actions (methods).

\section{Actions}

Component objects are active, this means that they can react to input
(like following a link or submitting a form) and they can take some
kind of action in response to the input. Actions are blocks of common
lisp code which get called when a user interacts with a component.

Actions generally do one of three things:

\begin{itemize}
\item Change the state (either GUI or system) of the application. For
  example: A validation component may simply set a good/bod flag in
  some other form component; or a login form component may simply set
  a few variables in the current session.
\item Create another component and let this new component do
  something. For example: A search component may collect the search
  criteria and then ask a result listing component to display the
  results; A result litsing component may ask an object editor
  component to display a form and allow the user to edit an object.
\item Return to the previous component. For example: The user may tell
  the editor component that they're done editing the object and the
  editor component can let the result listing component take over
  again.
\end{itemize}

Whenever one component, through an approiatly defined action, asks
another component to do something we say that the first component has
``CALLed'' the other. When the other component is done and wants to
return control to the first component we say that the other component
``ANSWERs''. Surprisingly enough \ucw{} provides to operators,
\sym{CALL} and \sym{ANSWER} for just this purpose. Example:

\begin{verbatim}
(CALL 'login)
(CALL 'welcome-page)
\end{verbatim}

\begin{verbatim}
(when (and (string= "marco" username)
           (string= "marco" password))
  (ANSWER t))
\end{verbatim}

\subsection{Actions and Forms}

In most web frameworks a form, or an href with parameters, puts the
various inputs in the request and then calls a ``handler'' (be it a
script a servelt or whatever) which takes, conceptually, one
parametetr: the request object and returns the response object. This
is not how \ucw{} actions are generally used (though fou can do this
if you wish). \ucw{} actions are methods which take parameters,
grabbing these parameters from the request object is handled by Vodoo
Majik.

\begin{verbatim}
(defaction login ((c login-component) username password)
  (when (and (string= "user" username)
             (string= "secret" password))
    (answer t)))

(defmethod render-on ((response response) (c login-component))
  (let ((username "")
        (password ""))
    (<ucw:form :action (login c username password)
      "Username: " (<ucw:text :accessor username)
      "Password: " (<ucw:text :accessor password)
      (<ucw:submit :value "Login"))))
\end{verbatim}

Note how the username and password strings are passed directly to the
login action. This makes it much easier to know what inputs a
particular action needs and it allows us to simplfy combining actions:

\begin{verbatim}
(defaction foo ((c component) ...)
  ...
  (login c username password)
  ...)
\end{verbatim}


\section{Bits 'n Pieces}

\subsection{Sessions}

A \ucw{} application is just like a regular program with function
calls and return statements. Like a regular program shared globals
are, generally, frowned upon. \ucw{} applications store objects in the
session about as much as a regular program stores state in global
variables.

\subsection{Request Query Parameters}

Most web applications are litered with statements whcih grab a string
from the incoming request object. \ucw{} applications don't do that,
they use web accessors and closures instead. This has 2 advantages: 1)
you needn't worry about chosing independent names for parameters when
combining multiple components into a single form and 2) you can pass
``live'' lisp objects to the client and back (though no object
marshalling ever actually occurse. The main disadvantage to this is
that we've added a layer of Vodoo Majik.

\section{A Typical UCW Application}

Every \UCW{} application lives inside a \ucw{} server which, generally
speaking, is composed of:

\begin{itemize}
\item A lisp system. Currently \ucw{} supports CMUCL, SBCL, OpenMCL,
  CLISP and Allegro.
\item A web server. \ucw{} does not, directly, understand the HTTP
  protocol. Currently \ucw{} can work with Araneida, Aserve (and
  portableaserv) and or Apache (through mod\_lisp).
\item The \UCW{} system itself.
\end{itemize}

Within a single \ucw{} installation we can have multiple, relativly
independant, \ucw{} applications. Each application consists of:

\begin{itemize}
\item An application object.
\item A collection of \textit{entry-points} which are the publicly
  accessable urls of the application. Users will start interacting
  with a \ucw{} application by requesting an entry-point's url.
\item A collection of \textit{component}s which provide the application's
  graphical user interface.
\item A collection of \textit{action}s which specify how the
  application reacts to user input.
\item The core application itself (everything not related to the GUI).
\end{itemize}

\subsection{The Application Object}

While an essential component of every \ucw{} app the application
object isn't really that interesting (until we start customizing
\ucw{} internals). It mainly serves as a container for the \ucw{}
infrastructure and deals with, among other things:

\begin{itemize}
\item Looking up TAL template files.
\item Telling the backend where to find static resources (stylesheets,
  .html files, images, etc.)
\item Managing all the user sessions.
\item Managing the entry-points.
\end{itemize}

See the documentation of the \sym{APPLICATION} class for more details
on what this object does.

\subsection{The Components}

Every application will have ``something'' which specifies how the GUI
looks and acts, these ``things'' are called
\textit{component}s. Component objects provide the application with
its user interface and control, through appropiately defined methods,
how the application reacts user input.

\subsection{Actions}

Every time the user clicks on a link or submits a form \ucw{} will
call an action (the obvious exception to this rule are
entry-points). An action is a CLOS method specialized on a component
which does ``stuff'' (whatever the application needs to do) and can,
should the need arise, transfer control to other components.

Actions can either pass control to another action, via the macro
\sym{CALL} or the method \sym{CALL-COMPONENT}, or they can return
control using the macro \sym{ANSWER} or the method
\sym{ANSWER-COMPONENT}.

\subsubsection{Entry Points}

Entry points are just like actions in that they can do ``stuff'' and
can pass control to components. The difference is that entry-points
are associated with urls as opposed to components and entry-points can
not \sym{ANSWER}.

\section{Call backs}

\subsection{MIME Parts}

[NB: This is only implemented with the mod\_lisp backend]

When using POST requests with a mime-type of url-encoded all the
values passed to callbacks will be strings. When using
multipart/form-data however we have a value (the body of the part) and
a set of mime headers.

\section{REQUEST-CONTEXTs}

Everything un \ucw{} happens within a particular request
context. Request contexts contains, among other things, the request
object (which is used to locate the proper action and callbacks), the
response object (which contains the stream to whcih output should go),
the current application object, etc.

\section{Forms}

\subsection{Accessing input values}

There are quite a few ways to get the values passed by a client in a
form:

The most common way is to use ucw's :accessor attributes. Accessors
can either be tied to lexcial variables defined in the render-on
method (in which case we lose backtracking) or to slots in the
component(s).

We can always just ignore ucw's callback machinery and use regular
input tags with named parameters and the get-parameter method.

\section{The Standard Component Library}

\ucw{} includes a library of generic components. All of the standard
components provide developers with some functionality (implemenented
via methods or actions) and most provide a default render-on method.
Generally developers will not want to use the standard components
directly but will subclass the components and customize them by
defining methods and actions. In particular the login and container
components must be subclassed to be usefull and all components should
be subclassed when changing the render-on method.

\begin{description}
\item[container] An component for allowing multiple components to
  share the same piece of GUI real estate.
\item[date-picker] A very simple date selection component.
\item[error] Showing errors and the associated back trace.
\item[login] A dialog asking the user for authentication information
  and validating it. The login component must be subclassed in order
  to be used.
\item[option-dialog] A multiple choice dialog.
\item[range-view] A component for showing a large set of data one
  sub-set at a time.
\item[tabbed-pane] A container which attempts to provide a GUI similar
  to the tabbed-panes found in other GUIs.
\item[ucw-inspector] An html object inspector based on SLIME's
  inspector.
\end{description}

For more details on how to use each component see the class
documentation.

\section{The Backend}

\ucw{} does not provide an http server but instead relies on a backend
to deal with the underlying HTTP issues. \ucw{} currently supports 3
different backends,

\begin{description}
\item[mod\_lisp] The mod\_lisp backend provides the neccessary code to
  a talk to a mod\_lisp enabled Apache server.
\item[aserve] Uses the portable version of franz's excellent aserve
  package.
\item[araneida] UCW can run inside an araneida v0.9 server. UCW
  applications and non-ucw applications can co-exist within the same
  araneida image (though they can't communicatie with each other).
\end{description}

Every \ucw{} server is tied to exactly one backend though applications
can, theoretically, be registered in multiple server. This particular
functionality is completely untested.

\section{Vodoo Majik}

In a couple of places in this article we have refered to \ucw{}'s
``Vodoo Majik,'' there is no Vodoo Majik in \ucw{}. No goats nor
virgins are required for running \ucw{} applications. What we mean by
Vodoo Majik is simply something that \ucw{} does for you without you
having to worry about how it works, the interested can (and probably
should) go and read the code to figure what's really going on.

\end{document}
